\documentclass[10pt]{report}
%\usepackage{fullpage}
\usepackage{graphicx}   % if you want to include graphics files
\usepackage{subfigure}
\usepackage{setspace}
\usepackage{amsxtra}
\usepackage{bm}

\usepackage{amsmath}
\usepackage{amsfonts}
\usepackage{amssymb}
\usepackage{multirow}
\usepackage{algorithmic}
\usepackage{algorithm}

% graphics path
\graphicspath{{Figs/}}
%\usepackage{rotating}

%turns all margins to 1in
\addtolength{\oddsidemargin}{-.875in}
\addtolength{\evensidemargin}{-.875in}
\addtolength{\textwidth}{1.75in}

\addtolength{\topmargin}{-.875in}
\addtolength{\textheight}{1.75in}

\DeclareMathOperator{\sgn}{sgn} 
\setcounter{secnumdepth}{2}
\author{John Swoboda}
\title{\textbf{Incoherent Scatter Radar Simulator}}

\begin{document}
\maketitle

\chapter{Introduction}

This report covers the physics, and signal processing used to create the incoherent scatter radar (ISR) simulator that is in the code base. 

\chapter{ISR Spectrum}
\section*{Overview}

This is a description of the ISR spectrum process describes the mathematical basis for the software code base ISRSpectrum. It is based off of the development in  \cite{kudeki:milla:1} and \cite{Kudeki:2006kx} to form a spectrum.

\section*{Determining Spectra}
The first step comes from \cite{kudeki:milla:1} where a lumped circuit model is used to describe the spectrum. In it the independent thermal fluctuations of each species of ions and electrons are treated as current sources and the macroscopic conductances are treated as discrete components. The electric field $E$ impinged from the radar acts as a voltage. This lumped circuit model, seen in Figure \ref{fig:circuit}, is derived is taking the scalar component of Ampere's law in the direction of $\mathbf{k}$.  

\begin{equation}
\label{eq:ampere}
-j\mathbf{k} \times \mathbf{H} = \mathbf{J} +j\omega \epsilon_0 \mathbf{E},
\end{equation}

\noindent which then yields,

\begin{equation} 
\label{eq:ampscaler}
0=(\sigma_i +\sigma_e)E +\frac{\omega}{k}e(n_{ti}-n_{te}) +j\omega \epsilon_0 E.
\end{equation}

\begin{figure}[!h]
\centering
\includegraphics[width=3.0in]{circuit}
% where an .eps filename suffix will be assumed under latex, 
% and a .pdf suffix will be assumed for pdflatex; or what has been declared
% via \DeclareGraphicsExtensions.
\caption{Lumped circuit model seen in  \cite{kudeki:milla:1}.}
\label{fig:circuit}
\end{figure}

\noindent Using the electron current expression, $-\omega k^{-1}en_e = E\sigma_e -\omega k^{-1}en_{te}$, these equations can be rearranged to solve for $n_e$, 

\begin{equation}
\label{eq:neeq}
n_e(\mathbf{k},\omega) =  \frac{(j\omega\epsilon_0 + \sigma_i) n_{te}(\mathbf{k},\omega)}{j\omega\epsilon_0 +\sigma_e+\sigma_i} + \frac{\sigma_en_{ti}(\mathbf{k},\omega)}{j\omega\epsilon_0 +\sigma_e+\sigma_i}.
\end{equation}

\noindent To determine the power spectrum we square and average Equation \ref{eq:neeq} taking into account that the terms $n_{te}$ and $n_{ti}$ are independent of one and other we result in the following
\begin{equation}
\label{mainspeceq}
\langle \left|n_e(\mathbf{k},\omega)\right|^2\rangle = \frac{|j\omega\epsilon_0 + \sigma_i|^2 \langle |n_{te}(\mathbf{k},\omega)|^2\rangle}{|j\omega\epsilon_0 +\sigma_e+\sigma_i|^2} + \frac{| \sigma_e|^2 \langle |n_{ti}(\mathbf{k},\omega)|^2\rangle}{|j\omega\epsilon_0 +\sigma_e+\sigma_i|^2}.
\end{equation}

We can generalize them for multiple ion species by simply summing over the thermal fluctuations and conductances in Equation \ref{eq:ampscaler},

\begin{equation} 
\label{eq:ampscalersum}
0=\left(\displaystyle \sum_k^K\sigma_{ik} +\sigma_e\right)E +\frac{\omega}{k}e\left(\sum_k^Kn_{tik}-n_{te}\right) +j\omega \epsilon_0 E.
\end{equation}

\noindent This then augments the power spectrum in Equation \ref{mainspeceq} to the following

\begin{equation}
\label{eq:sumspeceq}
\displaystyle \langle \left|n_e(\mathbf{k},\omega)\right|^2\rangle = \frac{\left|j\omega\epsilon_0 +  \sum_k^K\sigma_{ik} \right|^2 \langle |n_{te}(\mathbf{k},\omega)|^2\rangle}{\left|j\omega\epsilon_0 +\sigma_e+ \sum_k^K\sigma_{ik} \right|^2} + \frac{| \sigma_e|^2 \left \langle \left|\sum_k^Kn_{tik}(\mathbf{k},\omega)\right|^2\right\rangle}{\left|j\omega\epsilon_0 +\sigma_e+ \sum_k^K\sigma_{ik} \right|^2}.
\end{equation}

%%%%%%%%%%%%%%%%%%%%%%%%%%%%%%%%%%%%%%%%%%%%%%%%%%%%%%%%%%%%%%%%%%%%%%
\section*{Gordeyeve Integrals}
The power spectrum of the thermal fluctuation for each species $s$ can be determined by the following,

\begin{equation}
\label{eq:thermalfl}
\frac{\langle|n_{ts}(\mathbf{k},\omega)|^2\rangle}{N_0} = 2\text{Re}\{J_s(\omega_s)\},
\end{equation}

\noindent where $N_s$ is the average density for the species.  Also the conductance for each species $s$ can be determine from the following,

\begin{equation}
\label{eq:cond}
\frac{\sigma_{s}(\mathbf{k},\omega)}{j\omega\epsilon_0} = \frac{1-j\omega_s J_s(\omega_s)}{k^2\lambda_s}
\end{equation}

\noindent where $\omega_s \equiv \omega-\mathbf{k}\cdot\mathbf{V}_s $ is the Doppler shifted frequency and $\lambda_s \equiv \sqrt{\frac{\epsilon_0 KT_s}{N_s q_s^2}}$ is the Debye length for each species.

The $J_s$ terms can be represented as follows

\begin{equation}
\label{eq:gord}
J_s(\omega)\equiv \int_0^\infty \langle e^{j\mathbf{k}\cdot\Delta \mathbf{r}_s}\rangle e^{j\omega\tau}d\tau
\end{equation}

\noindent These terms are known as Gordeyeve integrals, which are the one sided Fourier transforms of the characteristic functions of the particle displacements $\langle e^{j\mathbf{k}\cdot\Delta\mathbf{r}_s}\rangle$.  

The particle displacement function can change depending on magnetic field and collisionality of the plasma. For the high latitude F-region in the ionosphere a case of general importance is one of a non-magnitized and collision less plasma, where $\Delta\mathbf{r} = \mathbf{v}\tau$ where $\tau$ is the time interval. Assuming a Maxwellian the PDF of one dimensional displacement is

\begin{equation}
\label{eq:pdfr}
f(\Delta r) = \frac{1}{\sqrt{2\pi \langle r^2 \rangle}}e^{\frac{-\Delta r^2}{2\langle r^2\rangle}}.
\end{equation}
 
\noindent The variance term $\langle r^2 \rangle$ can be represented as
\begin{equation}
\label{eq:var}
\langle r^2 \rangle = \langle v^2 \rangle \tau^2 = \frac{KT_s}{m_s} \tau^2
\end{equation}
 \noindent where $T_s$ is the temperature of the species, $K$ is Boltzmans constant and $m_s$ is the mass of the species in kg. To simplify notation like in \cite{kudeki:milla:1}, we will refer to $\sqrt{KT_s/m_s}$ as $C$. Which yields the following single particle ACF,
 
 \begin{equation}
\label{eq:pdfall}
\langle e^{j\mathbf{k}\cdot\Delta \mathbf{r}}\rangle= e^{-\frac{1}{2}k^2C^2 \tau^2}.
\end{equation}
 
 To model collisions we use the term $\nu$ as the collision frequency for the species. If $\nu<<kC$ then \ref{eq:pdfall} can be used as the single particle ACF. If not the following must be used.
 
 \begin{equation}
 \label{eq:colspacf}
 \langle e^{j\mathbf{k}\cdot\Delta \mathbf{r}}\rangle = e^{-\frac{k^2C^2}{\nu^2}\left( \nu \tau-1+e^{-\nu\tau}\right)}
 \end{equation}
 
Lastly if one is to add a magnetic field to the equations the single particle ACFs must now take into a account the orientation of the magnetic field. The authors of \cite{kudeki:milla:1} use the convention of breaking up the Bragg vector $\mathbf{k}$ into two components, one parallel to the magnetic field, $k_{\parallel}$ and one perpendicular,$k_{\perp}$, as such, $\mathbf{k}= \hat{b}k_{\parallel}+\hat{p}k_{\perp}$. This yields the following formulation for the single particle ACF,

 \begin{equation}
\label{eq:pdfmag}
\langle e^{j\mathbf{k}\cdot\Delta \mathbf{r}}\rangle= e^{-\frac{1}{2}k_{\parallel}^2C^2 \tau^2}\times e^{-\frac{2k_{\perp}^2C^2}{\Omega^2} \sin^2(\Omega\tau/2)},
\end{equation}

\noindent where the gyro frequency is $\Omega = qB/m$This formulation neglects the effects of collisions which if taken into account yields the following single particle ACF,

\begin{equation}
\label{eq:colspacf}
\langle e^{j\mathbf{k}\cdot\Delta \mathbf{r}}\rangle = e^{-\frac{k_\parallel^2C^2}{\nu^2}\left( \nu \tau-1+e^{-\nu\tau}\right)}\times e^{-\frac{k_\perp^2C^2}{\nu^2+\Omega^2}\left(\cos(2\gamma) + \nu \tau-e^{-\nu\tau}\cos(\Omega\tau-2\gamma)\right)},
\end{equation}
 
\noindent where $\gamma = \tan^{-1}(\nu\Omega)$. The for the case with the magnetic field as one gets closer to being fulling perpendicular to $\mathbf{B}$ the single particle ACFs become much more narrow band, to the point of becoming delta functions in the frequency space. It is necessary to use other methods beyond numerical integration to determine the Gordeyeve Integrals. The authors of \cite{kudeki:milla:2} get around this problem by making a PIC code to determine the particle statistics. 


%%%%%%%%%%%%%%%%%%%%%%%%%%%%%%%%%%%%%%%%%%%%%%%%%%%%%%%%%%%%%%%%%%%%%%
 \section*{Computational Considerations}
One of the main challenges to calculating the ISR spectrums is calculating the Gordeyeve integrals. The case with no collisions or magnetic fields can be done analytically using Dawsons integral. This can be done using the identity

\begin{equation}
\label{eq:daw1}
jZ(\theta) = \int_0^{\infty} e^{-j\theta t}e^{-\frac{t^2}{4}}dt = \sqrt{\pi}e^{-\theta^2}-j2e^{-\theta^2}\int_0^\theta e^{t^2}dt.
\end{equation}

\noindent Using the terms found in Equation \ref{eq:pdfall}, $\theta=\omega_s/\left(\sqrt{2}kC\right)$ and $t=\sqrt{2}kC\tau$.

For other cases where analytical calculation is not possible a numerical integration scheme from \cite{Ooi:2007jx} is used. It is also possible to use a Chirp-z based algorithm that is shown in \cite{Li:1991gr} from the experiences of the author the first technique converges faster. The technique used in \cite{Ooi:2007jx} changes the variable of integration for integrals of the following form,

\begin{equation}
\label{eq:Sommer}
I=\int_a^b f(z) dz.
\end{equation}

\noindent The technique changes the variable $z$ in the following way,

\begin{equation}
\label{eq:newz}
z = \frac{1}{2}(a+b)+\frac{1}{2} (b-a)\text{Erf}(g(t)),
\end{equation}

\noindent where $g(t)$ is a function that is choosen so  $g(t)\rightarrow\pm \infty$ as $t\rightarrow\pm \infty$ and $Erf(u)$ is 
\begin{equation}
\label{eq:erf1}
\text{Erf}(u) = \frac{2}{\sqrt{\pi}}\int_0^u e^{-t^2}dt.
\end{equation}

\noindent Discretizing and changing variables the integral in Equation \ref{eq:Sommer} becomes the following sum

\begin{equation}
\label{eq:erfsum1}
I=\displaystyle \sum_{n=-N}^N A_nf\left( \frac{1}{2}(a+b)+\frac{1}{2} (b-a)\text{Erf}(g(nh))\right)
\end{equation}

\noindent where,
\begin{equation}
\label{eq:anterm}
A_n = g'(nh)e^{-g(nh)^2}.
\end{equation}

\noindent Like in \cite{Ooi:2007jx}, $g(nh) = \sinh (nh)$ and the grid spacing $h$ is the following,

\begin{equation}
\label{eq:hterm}
h = \frac{1}{N}\ln(1.05\sqrt{2}N).
\end{equation} 


Lastly to avoid cases of divid by zero errors the main equations have to be rearrange slightly. First off because some ion species could have zero density Equation \ref{eq:cond} uses the Debye length of the electron species,$\lambda_e$ as follows

\begin{equation}
\label{eq:condnew}
\frac{\sigma_{s}(\mathbf{k},\omega)}{j\omega\epsilon_0} = \frac{1-j\omega_s J_s(\omega_s)}{k^2\lambda_e} \left(\frac{q_sT_eN_s}{q_eT_sN_e}\right).
\end{equation}

Also, to avoid having to more calculations then necessary the $j\omega\epsilon_0$. terms of Equation \ref{eq:sumspeceq} are moved around. Thus it becomes,

\begin{equation}
\label{eq:sumspeceqfinal}
\displaystyle \langle \left|n_e(\mathbf{k},\omega)\right|^2\rangle =  \frac{\left|1 +  \sum_k^K\frac{\sigma_{ik}}{j\omega\epsilon_0} \right|^2 \langle |n_{te}(\mathbf{k},\omega)|^2\rangle}{\left|1 +\frac{\sigma_e+ \sum_k^K\sigma_{ik}}{j\omega\epsilon_0} \right|^2} + \frac{\left| \frac{\sigma_e}{j\omega\epsilon_0} \right|^2\left \langle \left|\sum_k^Kn_{tik}(\mathbf{k},\omega)\right|^2\right\rangle}{\left|1 +\frac{\sigma_e+ \sum_k^K\sigma_{ik}}{j\omega\epsilon_0} \right|^2}.
\end{equation}

\section{Examples}
We can see in Figure \ref{fig:diffspectrums} examples of ISR spectrums from different ISR systems. The spectrums were generated using the the parameters values $N_e=1e11$, $T_e=3000^o$K and $T_i=3000^o$K and the system parameter values seen in Table \ref{tab:ISRsys}. The ion acoustic frequency$f_{ia}$ for each system with the following plasma parameters its wavelength $\lambda$ was calculated using the following formula,

\begin{equation}
\label{eq:iaf}
f_{ia} = \frac{\lambda}{2}\sqrt{\frac{k_bT_e +k_b\gamma_iT_i}{M}},
\end{equation}

\noindent where $M$ is the ion mass in kg, $k_b$ is Botlzmann's constant and $\gamma_i$ is the adiabatic index which is set to 3 in all cases.
\begin{figure}[!h]
\centering
\includegraphics[width=7.0in]{DifferentSystems}

\caption{Spectrums From Different ISR Systems}
\label{fig:diffspectrums}
\end{figure}

\begin{table}[!h]
\centering
\caption{ISR System Parameters}
\label{tab:ISRsys}
\begin{tabular}{lllll}
System Name & $f_0$ in MHz & $f_s$ in kHz & $\alpha$ in $^o$ &  \\
AMISR       & 449          & 50           & 70               &  \\
Sondrestrom & 1290         & 100          & 80               &  \\
Haystack    & 440          & 50           & 65               &  \\
Arecibo     & 430          & 50           & 45               &  \\
Jicamarca   & 50           & 10           & 1                & 
\end{tabular}
\end{table}
%%%%%%%%%%%%%%%%%%%%%%%%%%%%%%%%%%%%%%%%%%%Chapter 2%%%%%%%%%%%%%%%%%%%%%%%%%%%%%%%%%%%%%%%%%%%%%
\chapter{Data Formation}
The 3-D ISR simulator creates data by deriving a time filter from the autocorrelation functions and applying them to complex white Gaussian noise generators. Stating this in another way, every point in time and space has a noise plant and filter structure as in Figure \ref{fig:IQdiagram}. 
\begin{figure}[h!]
\centering
\includegraphics[width=4in]{diagrampart}
\caption{Diagram for I/Q simulator signal flow.}
\label{fig:IQdiagram}
\end{figure}

These filters are created using spatial averaging of an intrinsic ISR spectrum which come from the parameters at each point in space and space put into the simulator. The spatial averaging occurs by way of the antenna beam pattern, range gate size and pulse pattern. The beam pattern and range gate size act as an averaging across the range, azimuth and elevation dimensions of the intrinsic ISR spectrums. The pulse pattern acts as a windowing of the final samples of the data.

\begin{figure}[!h]
\centering
\includegraphics[width=7.0in]{diagram}
\caption{ISR Simulation Diagram}
\label{fig:isrdiag}
\end{figure}




The simulator discretizes the ionosphere into a set of $m\in 0,1,...M-1$ range gates that have specific values of electron density ($N_e$), ion temperature ($T_i$), electron temperature($T_e$), and masses of ion species ($M_1$ and ($M_2$)) that are present and their ratios ($\kappa$).  The simulator creates a filter by taking the square root of the IS spectrum derived from a model,

\begin{equation}
\label{eq1}
H_m(\omega) = \sqrt{S_m(\omega | \: N_e, T_i,  T_r, M_{i1}, M_{i2},\kappa)}.
\end{equation}

\noindent The filter can be created by taking the inverse Fourier transform of $H_m(\omega)$.  Complex white Gaussian noise  ($w(k)\sim CN(0,1)$) is then pushed though each of the filters and then windowed by the pulse creating the following:   

\begin{equation}
\label{eq2}
y_m (k)= s(k)\left[h_m(k)*w(k)\right],
\end{equation}
 
\noindent where $s(k)$ is the pulse shape.  Currently in the the filtering is implemented by using the frequency response of the filter and multiplying by the complex noise and then inverse Fourier transforming it back.  A diagram of a single noise plant can be seen in Figure \ref{fig:singlefilt}

\begin{figure}[!t]
\centering
\includegraphics[width=4in]{diagrampart}
% where an .eps filename suffix will be assumed under latex, 
% and a .pdf suffix will be assumed for pdflatex; or what has been declared
% via \DeclareGraphicsExtensions.
\caption{I/Q Simulator Diagram}
\label{fig:singlefilt}
\end{figure}

After the data for each range gate $y_m(k)$ is created the power of the return is calculated

\begin{equation}
\label{eq3}
P_r = \frac{cG \lambda^2}{2(4\pi)^2}\frac{P_t }{R^2}\frac{\sigma_e N_e}{(1+k^2\lambda_D^2),(1+k^2\lambda_D^2 + T_r)}
\end{equation}
 
 \noindent where $P_r$ is the power received, $c$ is the speed of light, $G$ is the gain of the antenna, $P_t$ is the power of the transmitter, $\sigma_e$ is the electron radar cross section, $k$ is the wavenumber of the radar, $\lambda_D$ is the Debye length and $T_r$ is the temperature ratio.\footnote{Currently all of the radar parameters used in the simulations are from the Poker Flat ISR system.}
 
Once the power has been calculated for each range all of the data is delayed and summed together so as to model the arrival of the radar return at the receive: 
 
\begin{equation}
\label{eq4}
x(n) = \displaystyle\sum\limits_{m =0}^{M-1} \alpha(m)y_m(n-m),
\end{equation}

\noindent where $\alpha(m) = \sqrt{P_r(m)}$.  A full diagram of the model can be seen in Figure \ref{fig:sumrule}.

\begin{figure}[!t]
\centering
\includegraphics[width=5in]{diagram}
% where an .eps filename suffix will be assumed under latex, 
% and a .pdf suffix will be assumed for pdflatex; or what has been declared
% via \DeclareGraphicsExtensions.
\caption{I/Q Simulator Diagram}
\label{fig:sumrule}
\end{figure}

The extension of this idea to three dimensional space takes the same concepts and repeats the process for multiple beams.  The main differences is how the data is that the plasma parameters are first represented in Cartesian space.  Each point will have a filter noise plant set up as seen in Figure \ref{fig:singlefilt}.  From there the data will be combined together with a weighting depending on the beam pattern.  The data will then be delayed and weighted further depending on the range the point is located.  A basic diagram can be seen in Figure \ref{fig:beamdia}.

\begin{figure}[!t]
\centering
\includegraphics[width=2in]{beamsampling}
% where an .eps filename suffix will be assumed under latex, 
% and a .pdf suffix will be assumed for pdflatex; or what has been declared
% via \DeclareGraphicsExtensions.
\caption{Beam Sampling Diagram}
\label{fig:beamdia}
\end{figure}


%%%%%%%%%%%%%%%%%%%%%%%%%%%%%%%%%%%%%%%%Chapter 3%%%%%%%%%%%%%%%%%%%%%%%%%%%%%%%%%%%%%%%%%%%%%%%%%%%
\chapter{ISR Processing}

After the IQ data has been created it is processed to create estimates of the ACF at desired points of space. This processing follows a flow chart seen in Figure \ref{fig:chain}.

\begin{figure}[!t]
\centering
\includegraphics[width=6in]{datastackchain}
\caption{ISR signal processing chain, with signal processing operations as squares and data products as diamonds.}
\label{fig:chain}
\end{figure}



\section{Lag Product Formation}
The lag product formation is an initial estimate of the autocorrelation function.  The sampled I/Q can be represented as $x(n) \in\mathbb{C}^N$ where $N$ is the number of samples in an inter pulse period.  At this point the first step in estimating the autocorrelation function is taken.  For each range gate $m\in 0,1,...M-1$ an autocorrelation is estimated for each lag of $l \in 0,1...,L-1$.  To get better statistics this operation is performed for each pulse $j\in 0,1,...J-1$ and then summed over the $J$ pulses.  The entire operation to form the initial estimate of $\hat{R}(m,l)$ can be seen in Equation \ref{lagpro}:

\begin{equation}
\label{lagpro}
\hat{R}(m,l) = \displaystyle\sum\limits_{j=0}^{J-1} x(m-\lfloor l/2\rfloor,j)x^*(m+\lceil l/2 \rceil,j).
\end{equation}

The case shown in Equation \ref{lagpro} is a centered lag product, other types of lag products calculations are available but generally a centered product is used. In the centered lag product case range gate index $m$ and sample index $n$ can be related by $m=n-\lfloor L/2\rfloor$ and the maximum lag and sample relation is $M=N-\lceil L/2 \rceil$.  This lag product formation is the first step in taking a discrete Wigner Distribution \cite{TFAcohen}.

\section{Summation Rules}
Applying a summation rule is usually the next step in creating an estimate of the autocorrelation function.  This is done to get a constant range ambiguity across all of the lags for long pulse experiment\cite{nygren1996}. 

The summation rule idea can be seen in Figure \ref{fig:sumrule}.  In the figure the image on the left is a basic representation of an ambiguity function of a long pulse.  Its mirrored on the right with red bars which would show the integration area under it so the ambiguity function will be of equal size in range.  

In the processing this is basically a summing of lags from different ranges.  The amount of summing is similar to what is shown in Figure \ref{fig:sumrule}.  There are a number of different summing rule each with their own trade offs \cite{nygren1996}.  

Lastly an estimate of the noise correlation is subtracted out of $\hat{R}(m,l)$, which is defined as $\hat{R}_w(m,l)$:

\begin{equation}
\label{lagpro}
\hat{R}_w(m,l) = \displaystyle\sum\limits_{j=0}^{J-1} w(m_w-\lfloor l/2\rfloor,j)w^*(m_w+\lceil l/2 \rceil,j),
\end{equation}

\noindent where $w(n_w)$ is the background noise process of the radar.  Often the noise process is sampled during a calibration period for the radar when nothing is being emitted.  The final estimate of the autocorrelation function after the noise subtraction and summation rule will be represented by $\hat{R}_f(m,l)$.
\begin{figure}[!t]
\centering
\includegraphics[width=3in]{sumrule}
% where an .eps filename suffix will be assumed under latex, 
% and a .pdf suffix will be assumed for pdflatex; or what has been declared
% via \DeclareGraphicsExtensions.
\caption{Summation Rule Diagram}
\label{fig:sumrule}
\end{figure}

\section{Nonlinear Least-Squares Fitting}
After the final estimation of the spectrum is complete the nonlinear least squares fitting takes place to determine the parameters.  The basic class of nonlinear least-squares problems as seen in \cite{kayvol1}, are shown in Equation \ref{nlls},

\begin{equation}
	\hat{\mathbf{p}}= \underset{\mathbf{p}}{\text{argmin}} (\mathbf{y}-\bm{\theta}(\mathbf{p}))\bm{\Sigma}(\mathbf{y}-\bm{\theta}(\mathbf{p}))^*.
\label{nlls}
\end{equation}

In Equation \ref{nlls}, the data represented as $\mathbf{y}$ would be the final estimate of the autocorrelation function $\hat{R}_f(m,l)$ at a specific range or its spectrum $\hat{S}_f(m,\omega)$.  The parameter vector $\mathbf{P}$ would be the plasma parameters $N_e$, $T_e$, $T_i$ and various other parameters including ion mass and ratios between different species.  The fit function $\bm{\theta}$ is the IS spectrum calculated from models, such as once seen in \cite{kudeki:milla:1}, smeared by the ambiguity function.  In the case of the long pulse the ambiguity can be simply applied by multiplying it with the autocorrelation function $R(l)$, if the summation rule is properly applied.  Lastly the correlation matrix $\bm{\Sigma}$ if often realized as a diagonal matrix for many ISR systems.  If it was not that would imply correlation with the measurement uncertainties between different lags or samples of the spectrum.  The diagonal values usually used, noted as $\sigma_i^2$, usually the same unless there is a larger measurement error for one of the lags or spectrums.  The following formula from  \cite{nicollsisrschool2013} can be used:

\begin{equation}
\label{sigpow}
\sigma_i = \frac{S}{\sqrt{J}}\left(1+\frac{1}{SNR}\right).
\end{equation}

\noindent where $S$ is the signal power and $SNR$ is the signal to noise ratio.  These quantities can be estimated using the calibration period.


In the past ISR researchers have used the Levenberg-Marquart algorithm to fit data \cite{nikoukar2008}.  This specific iterative algorithm moves the parameter vector $\mathbf{p}$ by a perturbation $\mathbf{h}$ at each iteration\cite{gavin:2013}.  Specifically Levenberg-Marquart was designed to be a sort of meld between two different methods Gradient Decent, and Gauss-Newton.  The perturbation vector $\mathbf{h}_{lm}$ can be calculated using the following:

\begin{equation}
\left[ \mathbf{J}^T\bm{\Sigma}\mathbf{J}\right]\mathbf{h}_{lm} =\mathbf{J}^T\bm{\Sigma}(\mathbf{y}-\bm{\theta})
\label{hlm}
\end{equation}

\noindent where $\mathbf{J}$ is the Jacobian matrix $\partial \bm{\theta}/\partial \mathbf{p}$ \cite{levenberg1944} \cite{marquardt:1963}


 \bibliographystyle{BibTeX/ieeetrsrt}

 \bibliography{BibTeX/litreview}

 \end{document}